\documentclass{article}

\usepackage[a4paper,landscape,margin=2cm]{geometry}
\usepackage{../../components/components}

\usepackage{fancyhdr}
\pagestyle{fancy}
\fancyhf{}
\fancyhead[L]{DL1 Math-Info}
\fancyhead[C]{Réduction des endomorphismes et matrices diagonalisables}
\fancyhead[R]{2024-2025}
\fancyfoot[L]{Ewen Rodrigues de Oliveira}
\fancyfoot[R]{\thepage}

\usepackage{paracol}

\begin{document}
\setlength{\columnsep}{2cm}

\begin{paracol}{2}

\exercise{1}{Sous-espaces stables}{
Soit \( f : \mathbb{R}^3 \to \mathbb{R}^3 \) défini par \( f(x, y, z) = (x + y, y + z, z) \).\\
Détermine si les sous-espaces suivants sont stables par \( f \) :
\begin{itemize}
  \item \( F_1 = \{(x, y, 0) \in \mathbb{R}^3\} \)
  \item \( F_2 = \text{Vect}(1, 0, -1) \)
  \item \( F_3 = \text{Vect}(1, 1, 1) \)
\end{itemize}
}

\exercise{2}{Valeurs propres et vecteurs propres}{
Soit \( f : \mathbb{R}^2 \to \mathbb{R}^2 \) défini par \( f(x, y) = (3x + y, x + 3y) \).\\
\begin{enumerate}
  \item Détermine la matrice de \( f \) dans la base canonique.
  \item Calcule le polynôme caractéristique.
  \item Trouve les valeurs propres et les vecteurs propres associés.
  \item Le spectre contient-il 0 ? \( f \) est-elle diagonalisable ?
\end{enumerate}
}

\exercise{3}{Polynôme caractéristique}{
Soit \( A = \begin{pmatrix} 2 & 0 & 0 \\ 1 & 3 & 0 \\ 4 & -1 & 2 \end{pmatrix} \).\\
\begin{enumerate}
  \item Calcule le polynôme caractéristique de \( A \).
  \item Détermine les valeurs propres de \( A \).
  \item Donne une base de chaque sous-espace propre.
  \item \( A \) est-elle diagonalisable ? Justifie.
\end{enumerate}
}

\switchcolumn

\exercise{4}{Étude complète d'une matrice}{
Soit \( A = \begin{pmatrix} 5 & 0 & 4 \\ 4 & 1 & 0 \\ -8 & 0 & -7 \end{pmatrix} \).\\
\begin{enumerate}
  \item Rappelle le polynôme caractéristique de \( A \) donné dans le cours.
  \item Trouve une base de chaque sous-espace propre.
  \item Déduis \( P \) et \( D \) telles que \( A = P D P^{-1} \).
  \item Vérifie que \( \mathbb{R}^3 = E_1(A) \oplus E_{-3}(A) \).
\end{enumerate}
}

\exercise{5}{Théorie — diagonalisabilité}{
Prouve que si \( f \in \mathcal{L}(E) \) est diagonalisable, alors :
\begin{itemize}
  \item Le polynôme caractéristique de \( f \) est scindé.
  \item Pour toute valeur propre \( \lambda \), la dimension de \( E_\lambda(f) \) est égale à sa multiplicité algébrique.
\end{itemize}
\textit{Indication : utilise le théorème du rang, le lien entre dimension et injectivité, et la décomposition en base de vecteurs propres.}
}

\exercise{6}{Diagonalisabilité}{
Soit \( A = \begin{pmatrix} 2 & 1 \\ 0 & 2 \end{pmatrix} \).\\
\begin{enumerate}
  \item Calcule son polynôme caractéristique.
  \item Donne ses valeurs propres et les sous-espaces propres.
  \item \( A \) est-elle diagonalisable ? Justifie rigoureusement.
\end{enumerate}
}

\switchcolumn
\newpage

\exercise{7}{Matrice à paramètres}{
Soit \( A = \begin{pmatrix} 1 & a \\ 0 & 1 \end{pmatrix} \) avec \( a \in \mathbb{R} \).\\
\begin{enumerate}
  \item Calcule le polynôme caractéristique.
  \item Pour quelles valeurs de \( a \) la matrice est-elle diagonalisable ?
  \item Interprète géométriquement le cas où \( a \neq 0 \).
\end{enumerate}
}

\end{paracol}

\end{document}
