\documentclass{article}

\usepackage[a4paper, left=2.5cm, right=2.5cm, top=3cm, bottom=3cm]{geometry}

\usepackage{../../components/components} % <-- ton fichier .sty, avec toutes tes définitions

\usepackage{fancyhdr}

% Configuration des en-têtes et pieds de page
\pagestyle{fancy}
\fancyhf{} % reset tout

\fancyhead[L]{DL1 Math-Info}
\fancyhead[C]{Applications linéaires}
\fancyhead[R]{2024-2025}

\fancyfoot[L]{Ewen Rodrigues de Oliveira}
\fancyfoot[R]{\thepage}

\begin{document}

\docTitle{Chapitre 4 : Applications linéaires et matrices associées}

\section{Introduction}

\subsection{Définitions}

Dans toute cette section, on notera \( E \) et \( F \) deux \(\mathbb{K}\)-espaces vectoriels, avec \(\mathbb{K} \in \left\{ \mathbb{R}, \mathbb{C} \right\} \).

\definition{Soit \( f : E \to F \) une application. On dit que \( f \) est une \textbf{application linéaire} si pour tous \( x, y \in E \) et tout \( \lambda, \mu \in \mathbb{K} \), on a : \textbf{\[f(\lambda x + \mu y) = \lambda f(x) + \mu f(y)\]}
On note \(\textcolor{primary}{\mathcal{L}(E, F)}\) l'ensemble des applications linéaires de \( E \) vers \( F \).
}

\remark{On a le résultat suivant : $f \in \mathcal{L}(E, F) \implies f(0) = 0$}
\vocabulary{
    \begin{itemize}
        \item $f \in \mathcal{L}(E, F)$ est un \textcolor{purple}{\textbf{isomorphisme}} \(\Leftrightarrow\) \(f\) est bijective.  On note \(\textcolor{purple}{Isom(E, F)}\) l'ensemble des isomorphismes de \(E\) vers \(F\).
        \item $f \in \mathcal{L}(E, F)$ est une \textcolor{purple}{\textbf{forme linéaire}} \(\Leftrightarrow\) \(F\) = $\mathbb{K}$
        \item $f \in \mathcal{L}(E, F)$ est un \textcolor{purple}{\textbf{endomorphisme}} \(\Leftrightarrow\) \(E\) = \(F\)
        \item Si $f$ est un endomorphisme et un isomorphisme, on dit que $f$ est un \textcolor{purple}{\textbf{automorphisme}} de \(E\), et on note \(\textcolor{purple}{f \in Aut(E)}\).
    \end{itemize}
  
}

\noindent{\text{On notera que si $f \in Isom(E, F)$, alors $dim(E) = dim(F)$.}}

\theorem{Proposition}{}{true}{
    $\mathcal{L}(E, F)$ est un sous-espace vectoriel de $F^E$ ($i.e.$ l'ensemble des applications de \(E\) vers \(F\)). 
}

\subsection{Noyau et image d'une application linéaire}

\definition{Soit \( f \in \mathcal{L}(E, F) \). On appelle \textbf{noyau} de \( f \) l'ensemble des éléments de \( E \) qui sont envoyés sur \( 0_F \) par \( f \) :
\[\textcolor{primary}{\ker(f) := f^{-1}(0)
}\]}
\training{Calculer le noyau de l'application linéaire \( f : \mathbb{R}^2 \to \mathbb{R}^3 \) définie par \( f(x, y) = (x + 2y, 3x - y, 4y) \).}
\carreaux{5}

\definition{On appelle \textbf{image} de \( f \) l'ensemble des éléments de \( F \) qui sont atteints par \( f \) :
\[\textcolor{primary}{\text{Im}(f) := f(E) }
\]}
\training{Calculer l'image de l'application linéaire \( f : \mathbb{R}^2 \to \mathbb{R}^3 \) définie par \( f(x, y) = (x + 2y, 3x - y, 4y) \).}
\carreaux{10}

\remark{On a le résultat suivant : $\ker(f)$  et $\text{Im}(f)$ sont respectivement des sous-espaces vectoriels de \(E\) et \(F\).}

\theorem{Proposition}{}{true}{
    \begin{itemize}
    \item $f \in \mathcal{L}(E, F)$ est \textbf{\textcolor{danger}{injective}} \(\Leftrightarrow\) \(\ker(f) = \{0_E\}\).

    \item $f \in \mathcal{L}(E, F)$ est \textbf{\textcolor{danger}{surjective}} \(\Leftrightarrow\) \(\text{Im}(f) = F\).
    \end{itemize}}

\vocabulary{
    $f \in \mathcal{L}(E, F)$ est dite \textcolor{purple}{\textbf{bijective}} si elle est injective et surjective \textit{(voir définition isomorphisme 1.A)}
}

\theorem{Proposition}{Aplication linéaire et bases}{true}{
    Si $E$ = $Vect(u_1, \ldots, u_n)$ alors $F$ = $Vect(f(u_1), \ldots, f(u_n))$.
}

\vocabulary{On appelle le \textcolor{purple}{\textbf{rang}} de $f$ et on note \(\textcolor{primary}{\text{rg}(f)}\) la dimension de l'image de \(f\) : }

\theorem{Théorème du rang}{}{true}{
    Soit \( f \in \mathcal{L}(E, F) \). On a la relation suivante :
    \[\textcolor{danger}{\text{dim}(\ker(f)) + \text{rg}(f) = \text{dim}(E)}
    \]
}
\training{Déduire des applications précédentes la dimension de \(\mathbb{R}^2\) est bien \(2\).}
\carreaux{3}
\theorem{Corollaire}{Injectivité et surjectivité}{true}{
    Si $f \in \mathcal{L}(E, F)$ telle que $dim(E) = dim(F)$, alors on a : \textcolor{danger}{$f$ injective \(\Leftrightarrow\) $f$ surjective \(\Leftrightarrow\) $f$ bijective.}
}

\subsection{Rapports géométriques}

\definition{Soient \( \lambda \in \mathbb{K} \), \( E = F \oplus G \) un \(\mathbb{K}\)-espace vectoriel.
    \begin{itemize}
        \item[a)] \( h \in \mathcal{L}(E) \) : \( h(x) = \lambda x \quad \forall x \in E \), est une \textcolor{primary}{homothétie de rapport \( \lambda \)}.

        \item[b)] \( p \in \mathcal{L}(E) \) : \( x + y \mapsto x \) 
        avec \( x \in F \), \( y \in G \), est la \textcolor{primary}{projection sur \( F \) parallèlement à \( G \)}.

        \item[c)] \( s \in \mathcal{L}(E) \) : \( x + y \mapsto x - y \)
        avec \( x \in F \), \( y \in G \),
        est la \textcolor{primary}{symétrie par rapport à \( F \) parallèlement à \( G \)}.
    \end{itemize}
}

\theorem{Proposition}{Comportements des rapports}{true}{
    Soit \( f \in \mathcal{L}(E, F) \) et \( g \in \mathcal{L}(F, G) \).
    \begin{itemize}
        \item[a)] Si \( \lambda = 0 \), alors \( f = 0 \) (l'application nulle).
        \item[ ] Si \( \lambda \neq 0 \), alors \( f \in \mathrm{Isom}(E, F) \),\\
        et son inverse est donné par : \( f^{-1} = \frac{1}{\lambda} \cdot \mathrm{Id}_E \).
        \item[b)] \( p \circ p = p \), alors :\\
        \hspace*{1em}\( \ker(p) = G \) et \( \operatorname{Im}(p) = F \).

        \item[c)] \( s \circ s = \mathrm{Id}_E \), donc :\\
            \hspace*{1em}\( \ker(s) = \{0\} \) et \( \operatorname{Im}(s) = E \),
    \end{itemize}
}

\text{\\}

\section{Matrices et applications linéaires}

\subsection{Matrices associées à une application linéaire}

\definition{
Soit \( f \in \mathcal{L}(E, F) \), \( B = (v_1, \ldots, v_n) \) une base de \( E \) et \( C = (w_1, \ldots, w_m) \) une base de \( F \). On appelle \textbf{matrice associée} de \( f \) (relative aux bases \( B \) et \( C \)) et on note
\[
\textcolor{primary}{\mathrm{Mat}_{B,C}(f)}
\]
la matrice dont les colonnes sont les coordonnées de \( f(v_i) \) dans la base \( C \), pour tout \( i \in \{1, \ldots, n\} \).
}

\example{
Soit \( f \in \mathcal{L}(\mathbb{R}^2, \mathbb{R}^3) \) définie par \( f(x, y) = (x + 2y, 3x - y, 4y) \).
Soit \( B = ((1, 0), (0, 1)) \) la base canonique de \( \mathbb{R}^2 \) et \( C = ((1, 0, 0), (0, 1, 0), (0, 0, 1)) \) la base canonique de \( \mathbb{R}^3 \).
\\
On calcule les vecteurs \( f(v_1) \) et \( f(v_2) \) :
\[
f(1, 0) = (1, 3, 0) = 1 \times (1, 0, 0) + 3 \times (0, 1, 0) + 0 \times (0, 0, 1)
\]
\[
f(0, 1) = (2, -1, 4) = 2 \times (1, 0, 0) - 1 \times (0, 1, 0) + 4 \times (0, 0, 1)
\]

La matrice associée de \( f \) relative aux bases \( B \) et \( C \) est donnée par :
\(
\textcolor{info}{\mathrm{Mat}_{B,C}(f) = \begin{pmatrix}
1 & 2 \\
3 & -1 \\
0 & 4
\end{pmatrix}}
\)
}

\vocabulary{On note $E^{*}$ et on appelle \textcolor{purple}{\textbf{dual de \(E\)}} l'ensemble des formes linéaires sur \(E\).}

\theorem{Proposition}{Base du dual}{true}{
    Soit \( B = (v_1, \ldots, v_n) \) une base de \( E \). On note \( B^* = (v_1^*, \ldots, v_n^*) \) la base duale de \( E^* \) définie par :
    \[
    v_i^*(v_j) = 
    \begin{cases}
        1 & \text{si } i = j \\
        0 & \text{sinon}
    \end{cases}
    \]
}
\theorem{Proposition}{Transposée de la matrice d'une application linéaire}{true}{
    Soit \( f \in \mathcal{L}(E, F) \) et \( B = (v_1, \ldots, v_n) \) une base de \( E \), \( C = (w_1, \ldots, w_m) \) une base de \( F \).\\
    On note \( B^* = (v_1^*, \ldots, v_n^*) \) la base duale de \( E^* \) et \( C^* = (w_1^*, \ldots, w_m^*) \) la base duale de \( F^* \).\\
    De plus, on note \({}^{t}f \in \mathcal{L}(F^*, E^*)\) l'application linéaire définie par :
    \[
    {}^{t}f(w^*) = w^* \circ f \quad \forall w^* \in F^*.
    \]

    La matrice associée de \({}^{t}f\) relative aux bases \( C^* \) et \( B^* \) est la transposée de la matrice associée de \( f \) relative aux bases \( B \) et \( C \) :
    \[
    \textcolor{danger}{\mathrm{Mat}_{C^*, B^*}({}^{t}f) = {}^{t}\mathrm{Mat}_{B, C}(f)}.
    \]
}


\subsection{Matrices de changement de base}

\vocabulary{Une matrice de changement de base se dit aussi \textcolor{purple}{\textbf{matrice de passage}}.}

\definition{Soit \( B = (v_1, \ldots, v_n) \) une base de \( E \) et \( C = (w_1, \ldots, w_n) \) une autre base de \( E \). On appelle \textbf{matrice de changement de base} de \( B \) vers \( C \) et on note
\[
\textcolor{primary}{\mathrm{P_{B \to C}} = \begin{pmatrix}
    \text{coord}_{B}(w_1) & \text{}_{B}(w_2) & \ldots & \text{coord}_{B}(w_n)
\end{pmatrix}}
\]

la matrice dont les colonnes sont les coordonnées des vecteurs de la base \( C \) exprimés dans la base \( B \).
}

\example{
% Dans R 2 , la matrice de passage de la base canonique à la base ( u , v ) avec u = ( 2 , 3 ) et v = ( 4 , 5 ) est : ( 2 4 3 5 ) . %

Soit \( B = ((1, 0), (0, 1)) \) la base canonique de \( \mathbb{R}^2 \) et \( C = ((2, 3), (4, 5)) \) une autre base de \( \mathbb{R}^2 \).\\

La matrice de changement de base de \( B \) vers \( C \) est donnée par :
\[
\textcolor{info}{\mathrm{P_{B \to C}} = \begin{pmatrix}
    2 & 4 \\
    3 & 5
    \end{pmatrix}}
    \]

}

\training{
    Soit \(w\) un vecteur de \(E\), \(X_1\) ses coordonnées dans la base \(B\), \(X_2\) ses coordonnées dans la base \(B_2\) et soit \(P_{B \to C}\) la matrice de passage de \(B\) à \(C\). Alors on a :

    \[
    \textcolor{success}{X_1 = P_{B \to C} \cdot X_2}
    \]
}

\remark{
    Une telle matrice de changement de base est toujours inversible, et son inverse est la matrice de changement de base de \( C \) vers \( B \).
}

\theorem{Théorème}{Formule de changement de base}{true}{
    Soit \( f \in \mathcal{L}(E, F) \), \( B = (v_1, \ldots, v_n) \) une base de \( E \) et \( C = (w_1, \ldots, w_m) \) une base de \( F \).\\
    Soit \( B' = (u_1, \ldots, u_n) \) une autre base de \( E \) et \( C' = (z_1, \ldots, z_m) \) une autre base de \( F \).\\
    On note \( P_{B \to B'} \) la matrice de changement de base de \( B \) vers \( B' \) et \( P_{C' \to C} \) la matrice de changement de base de \( C' \) vers \( C \).\\
    Alors :
    \[
    \textcolor{danger}{\mathrm{Mat}_{B', C'}(f) = P_{C' \to C} \cdot \mathrm{Mat}_{B, C}(f) \cdot P_{B \to B'}}
    \]
    avec $P_{C' \to C}$ = $P_{C \to C'}^{-1}$ .
}

\theorem{Corollaire}{Application aux endomorphismes}{true}{
    Soit \( f \in \mathcal{L}(E) \) un endomorphisme de \( E \). Soit \( B = (v_1, \ldots, v_n) \) une base de \( E \) et \( B' = (w_1, \ldots, w_n) \) une autre base de \( E \).\\
    On note \( P_{B \to B'} \) la matrice de changement de base de \( B \) vers \( B' \).\\
    Alors :
    \[
    \textcolor{danger}{\mathrm{Mat}_{B', B'}(f) = P_{B \to B'}^{-1} \cdot \mathrm{Mat}_{B, B}(f) \cdot P_{B \to B'}}
    \]
}

\end{document}