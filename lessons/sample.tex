\documentclass{article}

\usepackage[a4paper, left=2.5cm, right=2.5cm, top=3cm, bottom=3cm]{geometry}

\usepackage{../components/components} % <-- ton fichier .sty, avec toutes tes définitions

\usepackage{fancyhdr}

% Configuration des en-têtes et pieds de page
\pagestyle{fancy}
\fancyhf{} % reset tout

\fancyhead[L]{DL Math-Info}
\fancyhead[C]{Modèle type de document}
\fancyhead[R]{2025-2026}

\fancyfoot[L]{Ewen Rodrigues de Oliveira}
\fancyfoot[R]{\thepage}

\begin{document}

\docTitle{Modèle type de document}

\section{Page Composants}

\subsection{Définitions et Théorèmes}

Voici un exemple d’utilisation de tes composants :

\definition{Soit une suite \((u_n)_{n \in \mathbb{N}}\) quelconque. On dit que \((u_n)\) est décroissante si pour tout \(n \in \mathbb{N}\), on a \(u_n \geq u_{n+1}\).} 

\theorem{Théorème}{Pythagore}{true}{
Dans un triangle rectangle, le carré de l'hypoténuse est égal à la somme des carrés des deux autres côtés.
}

\vspace{1em}
\subsection{Carreaux}
\vspace{1em}

\noindent\textbf{Preuve :} \\[0.5em]
\hspace{\parindent}\carreaux{5}

\vspace{1em}
\subsection{Remarques de cours}
\vspace{1em}

\remark{Lorem ipsum dolor it amet, consectetur adipiscing elit.}
\attention{Lorem ipsum dolor sit amet, consectetur adipiscing elit.}
\illustration{Lorem ipsum dolor sit amet, consectetur adipiscing elit.}
\example{Lorem ipsum dolor sit amet, consectetur adipiscing elit.}
\vocabulary{Lorem ipsum dolor sit amet, consectetur adipiscing elit.}
\training{Lorem ipsum dolor sit amet, consectetur adipiscing elit.}

\end{document}
